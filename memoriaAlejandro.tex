\documentclass[a4paper]{article}
\usepackage{indentfirst}
\usepackage[utf8]{inputenc}
\usepackage[spanish]{babel}
\usepackage{hyperref}
\usepackage{url}

\title{CRA PECL2}
\author{Gorgues Valenciano, Alejandro\\
        \texttt{alegorval@gmail.com}
        \and
        Junquera Sánchez, Javier\\
        \texttt{javier.junquera.sanchez@gmail.com}}
}

\date{\today}

\begin{document}
    \maketitle

    \pagenumbering{arabic}

    \section*{Introducción}

    Se pide realizar un programa que, utilizando \emph{Prolog}, estudie un conjunto de frases a través de su gramática y las traduzca. Hemos dividido la pŕactica en dos partes: \textbf{4. Estructura sintáctica} y \textbf{5. Traductor sencillo}. Para coordinarnos en el trabajo, hemos creado un repositorio público en la plataforma \emph{GitHub}\cite{REPO}.

    \section*{Estructura sintáctica}

    Para el estudio de la estructura sintáctica, hemos completado el diccionario del ejemplo y las reglas gramáticas. Para construir la estructura propiamente dicha (los parámetros que recibirá \emph{draw.pl}) hemos parametrizado las reglas gramaticales y hemos ``generalizado" la forma de detectar las palabras del diccionario. Por ejemplo, para la frase \textbf{``El gato cazó un ratón"}, obtenemos un resultado como el siguiente:\\

    \obeyspaces{
        \texttt{oracion(X, [el, gato, cazo, un, raton],[]), draw(X).}\\

        \texttt{                   o}

        \texttt{                   |}

        \texttt{     +------------------+}

        \texttt{     gn                 gv}

        \texttt{     |                  |}

        \texttt{ +------+       +----------+}

        \texttt{det     n       v          gn}

        \texttt{ |      |       |          |}

        \texttt{ |      |       |      +-------+}

        \texttt{ |      |       |     det      n}

        \texttt{ |      |       |      |       |}

        \texttt{ |      |       |      |       |}

        \texttt{ el    gato    cazo    un    raton}\\

        \texttt{X = o(gn(det(el), n(gato)), gv(v(cazo), gn(det(un), n(raton))))}
    }

    \section*{Traductor sencillo}
    \subsection*{Subsection}
 Para la realización del ejercicio se han encontrado varios obstáculos, de los cuales algunos se han podido resolver, y otros no, es por ello que acontinuación se desarrollarán todos los problemas y soluciones obtenidas:
 1. Reglas gramaticales aplicadas en español:
    
 1.1*Se ha propuesto añadir la regla  "CODIGO" oracion(o(GV)) --> g_verbal(GV, _)."CODIGO" que      trata a las oraciones con conjunciones, de tal manera, que permite empezar una oración con un      verbo
    
 1.2*Se ha comprobado a partir de las oraciones a analizar, que una oración puede empezar con un     determinante, seguido de un nombre y un adjetivo, para ello se ha construido la siguiente regla     "CODIGO"g_nominal(gn(D, N, A), G, NUM) --> determinante(D, G, NUM), nombre(N, G, NUM), adjetivo       (A, G, NUM). "CODIGO"
    
 1.3*En el grupo nominal, también es posible encontrar que está formada por una proposición, en este     caso, puede ir acompañado de un determinante, nombre, o ambas, para ello se han construido las     siguientes reglas gramaticales: 
            1.3.1*"CODIGO"g_nominal(gn(D, GP), G, NUM) --> determinante(D, G, NUM), g_proposicional            (GP)."CODIGO"
            1.3.2*"CODIGO"g_nominal(gn(N, GP), G, NUM) --> nombre(N, G, NUM), g_proposicional(GP)."CODIGO"
            1.3.3*"CODIGO"g_nominal(gn(D, N, GP), G, NUM) --> determinante(D, G, NUM), nombre(N, G, NUM),               g_proposicional(GP)."CODIGO"
     
 1.4*Del mismo modo que se pueden encontrar proposiciones acompañadas de determinantes, o nombres,      con las preposiciones ocurre lo mismo, y adicionalmente, es posible  que se encuentre seguido de      un adjetivo como se ve a continuación:
            1.4.1*"CODIGO" g_nominal(gn(D, N, A, GP), G, NUM) --> determinante(D, G, NUM), nombre(N, G,             NUM),adjetivo(A, G, NUM), g_preposicional(GP)."CODIGO"
            1.4.2*"CODIGO"g_nominal(gn(D, N, GP), G, NUM) --> determinante(D, G, NUM), nombre(N, G, NUM),             g_preposicional(GP)"CODIGO".
     
 1.5*La preposición puede encontrarse también en el grupo verbal acompañado con un verbo, siendo en       este caso: "CODIGO"g_verbal(gv(V, GP), NUM) --> verbo(V,NUM), g_preposicional(GP). "CODIGO"
     
 1.6*En el grupo verbal, se ha tenido en cuenta en varios casos que una oración sea compuesta, es         decir, posea una conjunción que una dos oraciones, dicha conjunción puede ir seguida por un           verbo, un grupo nominal y verbo, o un adjetivo y un verbo como se puede ver:
            1.6.1*"CODIGO"g_verbal(gv(V, GN, CMP), NUM) --> verbo(V,NUM), g_nominal(GN, _, _), compuesta               (CMP)."CODIGO"
            1.6.2*"CODIGO"g_verbal(gv(V, A, CMP), NUM) --> verbo(V,NUM), adjetivo(A, _, NUM), compuesta                (CMP)."CODIGO"
            1.6.3*"CODIGO"g_verbal(gv(V, CMP), NUM) --> verbo(V,NUM), compuesta(CMP)."CODIGO"
     
 1.7*Como caso adicional en el grupo verbal, es posible que un verbo o un adjetivo se encuentre           seguido de un adverbio dadas las frases a analizar pedidas usando las siguientes reglas               gramaticales:
            1.7.1*"CODIGO"g_verbal(gv(V, A, GA), NUM) --> verbo(V,NUM), adjetivo(A, _, NUM), g_adverbial               (GA)."CODIGO"
            1.7.2*"CODIGO"g_verbal(gv(V, GA), NUM) --> verbo(V,NUM), g_adverbial(GA)."CODIGO"
    
 1.8*En el caso de las proposiciones, solo es necesario emplear la regla gramatical "CODIGO"      g_proposicional(gp(PN, GV)) --> pronombre(PN), g_verbal(GV, _)."CODIGO", ya que en las oraciones      pedidas sólo se encuentra dicha regla 
     
 1.9*Para las preposiciones se ha tenido en cuenta que pueden ir seguidos de un determinante, verbo, o nombre, para ello, se ha llamado a los grupos nominales y verbales con las siguientes reglas:
            1.9.1*"CODIGO"g_preposicional(gp(P, GN)) --> preposicion(P), g_nominal(GN, _, _)."CODIGO"
            1.9.2*"CODIGO"g_preposicional(gp(P, GV)) --> preposicion(P), g_verbal(GV, _)."CODIGO"
     
 1.10*Con los adverbios se ha planteado varios casos, en el primero de ellos, la oraciónt ermina con      el adverbio, en el segundo, temrina con un adverbio seguido de un adjetivo, en el tercero,           porsigue con un determinante o nombre llamando al grupo nominal, y en el cuerto, llama al grupo       verbal:
            1.10.1*"CODIGO"g_adverbial(ga(ADV)) --> adverbio(ADV)"CODIGO".
            1.10.2*"CODIGO"g_adverbial(ga(ADV, A)) --> adverbio(ADV), adjetivo(A, _, _)."CODIGO"
            1.10.3*"CODIGO"g_adverbial(ga(ADV, A, GN)) --> adverbio(ADV), adjetivo(A, G, NUM), g_nominal            (GN, G,NUM)."CODIGO"
            1.10.4*"CODIGO"g_adverbial(gp(ADV, GN)) --> adverbio(ADV), g_nominal(GN, _, _)."CODIGO"
            1.10.5*"CODIGO"g_adverbial(gp(ADV, GV)) --> adverbio(ADV), g_verbal(GV, _)."CODIGO"
     
 1.11*Para finalizar con las reglas empleadas, se ha tratado la divisón de dos oraciones como una          oración compuesta en la que se ha empleado la regla "CODIGO"compuesta(cnj(CNJ, O)) -->                conjuncion(CNJ), oracion(O)."CODIGO", donde la conjunción se encuentra seguida de una llamada a       la oración, pudiendo comenzar con un verbo o no
 2.Reglas gramaticales aplicadas en ingles:
   Con respecto a las reglas gramaticales en ingles, se ha empleado las mismas reglas gramaticales    que en el de español, con una diferencia en el grupo nominal, al tratarse de lengua inglesa, los    adjetivos se encuentran siempre por delante de los nombres, es por ello, que todas las reglas    gramaticales nominales que posean un nombre y un adjetivo, deberán cambiarse de orden como se      muestran a continuación:
     
 2.1*"CODIGO"nominal_g(gn(N, A), G, NUM) --> adjetive(A, G, NUM, _), noun(N, G, NUM, _)."CODIGO"
     
 2.2*"CODIGO"nominal_g(gn(D, N, A), G, NUM) --> determ(D, CV, NUM), adjetive(A, G, NUM, CV), noun(N,      G, NUM,_)."CODIGO"
     
 2.3*"CODIGO"nominal_g(gn(D, N, A, GP), G, NUM) --> determ(D, CV, NUM), adjetive(A, G, NUM, CV),          noun(N, G,NUM, _), prepositional_g(GP)."CODIGO"
3. Grupos gramaticales implementados tanto en ingles como en español:
Tanto en inglés como en español se han creado los mismos grupos gramaticales, los cuales, se explicarán a continuación:
     
3 .1*En primer caso, se ha decidido crear un grupo gramatical destinado a las proposiciones, llamado      g_proposicional, o propositional_g ya que en casos como "TEXTO""El hombre que vimos ayer es mi      vecino""TEXTO", se encuentra la proposición 'que', y es por ello que ha sido necesario crearlo
     
 3.2*En segundo caso, se ha creado el grupo preposicional g_preposicional, o , prepositional_g dado       que en las frases dadas, aparecen preposiciones, como es el caso de la frase "TEXTO" "Juan           estudia en la Universidad""TEXTO" con la preposición 'en'
     
 3.3*En tercer caso, se ha procedido a la creación del grupo gramatical g_adverbial, o adverbial_g,      ya que, se encuentran adverbios tanto de tiempo, como de cantidad, como es el caso de la frase      "TEXTO"El hombre que vimos ayer es mi vecino "TEXTO" con el adverbio de tiempo 'ayer', o      'yesterday'
     
 3.4*En último caso, se encuentran las oraciones compuestas, para ello, se ha decidido crear el      grupo gramatical compuesta, o complex, el cual comprobará si existe la conjunción 'y', o "and"        para proceder a clasificarla como una oración compuesta, un claro ejemplo es "TEXTO" "Juan es         delgado y María es alta.""TEXTO"

4. Frases traducidas correctamente
   Actualmente, debido al desconocimiento del entorno en las frases, hay dos de ellas, en las que ha    varios elementos que son utilizados en varias, no se llega a distinguir su uso, para ello se ha    decidido tratarlas de la manera más simple de modo que se muestren todas las frases traducidas    correctamente, como se puede ver en la práctica

5. Limitaciones de traducción 
   A la hora de realizar la traducción, se han encontrador varios problemas que se han intentado resolver. A continuación se detallarán cada uno de los problemas, y las soluciones tomadas, si se ha realizado:
   
 5.1*Para traducir nombres compuestos por apellidos, o títulos, se ha planteado que la traducción    tanto al español como al ingles sea como un único nombre, de tal manera que en la oración    "TEXTO"'oscar wilde escribio el fantasma de canterville'"TEXTO", en el español se ha tratado en el    diccionario como "CODIGO"nombre(n(n_18), m, sn) --> [oscar, wilde]. "CODIGO", y en el diccionario    en inglés como "CODIGO"noun(n(n_18), m, sn, v) --> [oscar, wilde]. "CODIGO".
   
 5.2*Cuando un nombre en inglés empieza por consonante o vocal y se encuentra acompañado por un    determinante indefinido, éste, debe decidir que tipo de determinante añadir, si "TEXTO"'a'"TEXTO",    o "TEXTO" 'an'"TEXTO" , para ello, se ha decidido emplear una nueva comprobación tanto en los       grupos gramaticales de nombre, adjetivo y detemrinante en inglés, como en español, de tal manera      que cuando el nombre empieza por una consonante, el determinante debe ser "TEXTO" 'a'"TEXTO" , y    cuando empieza en vocal, debe ser "TEXTO" 'an'"TEXTO", por ejemplo, si la oración posee el nombre     "TEXTO"'dog'"TEXTO" que corresponde en el diccionario "CODIGO"noun(n(n_12), m, sn, c) --> [dog].     "CODIGO, donde "TEXTO" 'c' "TEXTO" corresponde a la palabra consonante, si se encuentra acompañado     de un determinante indefinido, será necesario que se utilize "TEXTO" 'a' "TEXTO" con el registro     "CODIGO"determ(det(det_2), c, sn) --> [a].2 "CODIGO" en el diccionario en inglés
  
 5.3* La diferenciación entre "TEXTO"'big'"TEXTO" y "TEXTO"'large'"TEXTO" en las oraciones dadas no es    muy clara, debido al poco contexto que se muestran, es por ello que se comprobado mediante el uso    de internet, que en casos como estos, no hay ningún problema en utilizar "TEXTO"'big'"TEXTO", o      "TEXTO"'large'"TEXTO", de tal manera que se desconoce una solución correcta para su impementación
  
 5.4* Del mismo caso que se ha hablado de "TEXTO"'big'"TEXTO" y "TEXTO"'large'"TEXTO", ha ocurrido el mismo caso con las palabras "TEXTO"'in'"TEXTO" y "TEXTO"'at'"TEXTO", ya que para su correcta utilización, es necesario conocer el contexto debido a que "TEXTO"'in'"TEXTO" se aplica en unas circunstancias, y "TEXTO"'at'"TEXTO" en otras, es por ello que se ha empleado el uso únicamente de "TEXTO"'in'"TEXTO" para las preposiciones
  
 5.5*Debido a que ciertas traducciones no son correctas, se ha procedido a realizar de   manera literal dichas traducciones, como por ejemplo la frase "TEXTO"Ellos comen manzanas "TEXTO",   traducida al inglés correspondería con "TEXTO"They eat apples "TEXTO", sin embargo, se pide   traducirla a "TEXTO"They eat some apples"TEXTO", es por ello, que en el diccionario, se ha     procedido a convertir el nombre "TEXTO" apples"TEXTO" por "CODIGO"noun(n(n_6), f, pl, c) --> [some,   apples]. "CODIGO"

  5.6*Por último, en español, hay ciertas palabras que se sobreentienden, y es por ello que no es necesario añadirlas, sin embargo, en inglés, se deben añadir, es el caso de la frase "TEXTO" El hombre que vimos ayer es mi vecino"TEXTO", en inglés corresponde con "TEXTO" The man that we saw yesterday is my neighbour"TEXTO", en este caso, la palabra "TEXTO" vimos"TEXTO" elimina el determinante "TEXTO" nosotros"TEXTO", sin embargo en inglés se debe añadir, es por ello, que en el diccionario en ingles, se ha tenido que solucionar de la siguiente manera "CODIGO" verb(v(v_14), pl) --> [we, saw]. "CODIGO"

    Subsection P1.\\

    Subsection P2.

    \subsubsection*{Sub Subsection}

    \begin{itemize}
        \item \textbf{Section}
        \begin{enumerate}
            \item \emph{``Subsection"}
            \begin{enumerate}
                \item ``Sub Subsection"
            \end{enumerate}
        \end{enumerate}
    \end{itemize}

    \bibliographystyle{plain}
    \bibliography{bibliografia}
\end{document}
