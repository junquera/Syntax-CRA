\documentclass[a4paper]{article}
\usepackage{indentfirst}
\usepackage[utf8]{inputenc}
\usepackage[spanish]{babel}
\usepackage{hyperref}
\usepackage{url}

\title{CRA PECL2}
\author{Gorgues Valenciano, Alejandro\\
        \texttt{alegorval@gmail.com}
        \and
        Junquera Sánchez, Javier\\
        \texttt{javier.junquera.sanchez@gmail.com}}
}

\date{\today}

\begin{document}
    \maketitle

    \pagenumbering{arabic}

    \section*{Introducción}

    Se pide realizar un programa que, utilizando \emph{Prolog}, estudie un conjunto de frases a través de su gramática y las traduzca. Hemos dividido la pŕactica en dos partes: \textbf{4. Estructura sintáctica} y \textbf{5. Traductor sencillo}. Para coordinarnos en el trabajo, hemos creado un repositorio público en la plataforma \emph{GitHub}\cite{REPO}.

    \section*{Estructura sintáctica}

    Para el estudio de la estructura sintáctica, hemos completado el diccionario del ejemplo y las reglas gramáticas. Para construir la estructura propiamente dicha (los parámetros que recibirá \emph{draw.pl}) hemos parametrizado las reglas gramaticales y hemos ``generalizado" la forma de detectar las palabras del diccionario. Por ejemplo, para la frase \textbf{``El gato cazó un ratón"}, obtenemos un resultado como el siguiente:\\

    \obeyspaces{
        \texttt{oracion(X, [el, gato, cazo, un, raton],[]), draw(X).}\\

        \texttt{                   o}

        \texttt{                   |}

        \texttt{     +------------------+}

        \texttt{     gn                 gv}

        \texttt{     |                  |}

        \texttt{ +------+       +----------+}

        \texttt{det     n       v          gn}

        \texttt{ |      |       |          |}

        \texttt{ |      |       |      +-------+}

        \texttt{ |      |       |     det      n}

        \texttt{ |      |       |      |       |}

        \texttt{ |      |       |      |       |}

        \texttt{ el    gato    cazo    un    raton}\\

        \texttt{X = o(gn(det(el), n(gato)), gv(v(cazo), gn(det(un), n(raton))))}
    }

    \section*{Traductor sencillo}
    \subsection*{Subsection}

    Subsection P1.\\

    Subsection P2.

    \subsubsection*{Sub Subsection}

    \begin{itemize}
        \item \textbf{Section}
        \begin{enumerate}
            \item \emph{``Subsection"}
            \begin{enumerate}
                \item ``Sub Subsection"
            \end{enumerate}
        \end{enumerate}
    \end{itemize}

    \bibliographystyle{plain}
    \bibliography{bibliografia}
\end{document}
